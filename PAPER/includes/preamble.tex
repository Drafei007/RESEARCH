%---- PREAMBLE ----%

% Allows separate files to be used in the main file
\usepackage{subfiles}
\usepackage{natbib}
\bibpunct{(}{)}{;}{a}{,}{,}

% Algorithms
\usepackage{algorithm}
\usepackage[noend]{algpseudocode}
\renewcommand{\algorithmicrequire}{\textbf{Input:}}
\renewcommand{\algorithmicensure}{\textbf{Output:}}
\newcommand{\algorithmicbreak}{\State \textbf{break}}
\newcommand{\Break}{\algorithmicbreak}
\renewcommand{\algorithmicreturn}{\State \textbf{return}}
\newcommand{\algorithmicto}{\textbf{ to }}
\newcommand{\To}{\algorithmicto}
\newcommand{\algorithmictrue}{\textbf{true}}
\newcommand{\True}{\algorithmictrue}
\newcommand{\algorithmicfalse}{\textbf{false}}
\newcommand{\False}{\algorithmicfalse}
%\newcommand{\IndStatex}[1][1]{\Statex\hspace{6mm}}
%\newcommand{\IndIndStatex}[1][1]{\Statex\hspace{12mm}}

% Contains advanced math extensions
\usepackage{amsmath}

% Introduces the *proof* environment and the \theoremstyle command
%\usepackage{amsthm}

% Adds new symbols to be used in math mode, e.g. \mathbb
\usepackage{amssymb}

% To declare multiple authors
\usepackage{authblk}

% Provides extra comands as well as optimisation for producing tables
\usepackage{booktabs}

% Allows customisation of appearance and placements for figures/tables etc.
\usepackage{caption}

% Adds support for arbitrarily-deep nested lists
\usepackage{enumitem}

% Improves the interface for defining floating objects such as figures/tables
\usepackage{float}

\usepackage{fullpage}

% For easy management of document margins and the document page size
%\usepackage{geometry}

% Allows insertion of graphic files within a document
\usepackage{graphicx}

% Manage links within the document or to any URL when you compile in PDF
\usepackage{hyperref} 
\usepackage[dvipsnames]{xcolor}
\hypersetup{
	colorlinks,
	linkcolor={Rhodamine},
	citecolor={green!50!black},
	urlcolor={blue!80!black}
}

% Successor of amsmath
\usepackage{mathtools}

% No indentation, space between paragraphs
\usepackage{parskip}

%Include standalone .tex files
\usepackage{standalone}

% Define multiple floats (figures/tables) within one environment with individual captions 1a, 1b etc
\usepackage{subcaption}

\usepackage{tabularx}

\usepackage{tikz}

\usepackage{wrapfig}

\DeclarePairedDelimiter{\floor}{\lfloor}{\rfloor}
\DeclarePairedDelimiter{\ceil}{\lceil}{\rceil}

%Theorem style
%\theoremstyle{plain}% default
%\newtheorem{theorem}{Theorem}
%\newtheorem{corollary}{Corollary}
%\newtheorem{definition}{Definition}

%\theoremstyle{definition}
%\newtheorem{proposition}{Proposition}
%\newtheorem{exmp}{Example}[section]