\documentclass[oribibl]{llncs}

%---- PREAMBLE ----%

% Allows separate files to be used in the main file
\usepackage{subfiles}
\usepackage{natbib}
\bibpunct{(}{)}{;}{a}{,}{,}

% Algorithms
\usepackage{algorithm}
\usepackage{algpseudocode}
\renewcommand{\algorithmicrequire}{\textbf{Input:}}
\renewcommand{\algorithmicensure}{\textbf{Output:}}
\newcommand{\IndStatex}[1][1]{\Statex\hspace{6mm}}
\newcommand{\IndIndStatex}[1][1]{\Statex\hspace{12mm}}
\newcommand{\IndIndIndStatex}[1][1]{\Statex\hspace{18mm}}
\newcommand{\IndIndIndIndStatex}[1][1]{\Statex\hspace{24mm}}
\newcommand{\IndIndIndIndIndStatex}[1][1]{\Statex\hspace{30mm}}
\newcommand{\IndIndIndIndIndIndStatex}[1][1]{\Statex\hspace{36mm}}

% Contains advanced math extensions
\usepackage{amsmath}

% Introduces the *proof* environment and the \theoremstyle command
%\usepackage{amsthm}

% Adds new symbols to be used in math mode, e.g. \mathbb
\usepackage{amssymb}

% To declare multiple authors
\usepackage{authblk}

% Provides extra comands as well as optimisation for producing tables
\usepackage{booktabs}

% Allows customisation of appearance and placements for figures/tables etc.
\usepackage{caption}

% Adds support for arbitrarily-deep nested lists
\usepackage{enumitem}

% Improves the interface for defining floating objects such as figures/tables
\usepackage{float}

\usepackage{fullpage}

% For easy management of document margins and the document page size
%\usepackage{geometry}

% Allows insertion of graphic files within a document
\usepackage{graphicx}

% Manage links within the document or to any URL when you compile in PDF
\usepackage{hyperref} 
\usepackage{xcolor}
\hypersetup{
	colorlinks,
	linkcolor={red!50!black},
	citecolor={green!30!black},
	urlcolor={blue!80!black}
}

% Successor of amsmath
\usepackage{mathtools}

% No indentation, space between paragraphs
\usepackage{parskip}

%Include standalone .tex files
\usepackage{standalone}

% Define multiple floats (figures/tables) within one environment with individual captions 1a, 1b etc
\usepackage{subcaption}

\usepackage{tikz}

\usepackage{wrapfig}

\DeclarePairedDelimiter{\floor}{\lfloor}{\rfloor}
\DeclarePairedDelimiter{\ceil}{\lceil}{\rceil}

%Theorem style
%\theoremstyle{plain}% default
%\newtheorem{theorem}{Theorem}[section]
%\newtheorem{corollary}{Corollary}[theorem]

%\theoremstyle{definition}
%\newtheorem{defn}{Definition}[section]
%\newtheorem{proposition}{Proposition}[defn]
%\newtheorem{exmp}{Example}[section]

\title{Heuristics for the Score-Constrained Bin-Packing Problem}
\author{Asyl L. Hawa \and Rhyd M. R. Lewis \and Jonathan M. Thompson}
\institute{School of Mathematics, Cardiff University, Senghennydd Road, Cardiff, UK, CF24 4AG}
\date{\today}

\begin{document}
\maketitle

\begin{abstract}
	Abstract
\end{abstract}


\section{Introduction}
\label{sec:intro}

The one-dimensional bin packing problem (BPP) is a combinatorial optimisation problem that has been widely researched and discussed due to its ability to model a variety of real-life sitations.
The BPP can be described as follows: given a set of $n$ items of varying sizes $w$, and a finite number of bins with equal capacity $c$, find the minimum number of bins required to contain all of the items. As shown by \cite{garey1979}, this problem is NP-hard, and therefore (under the assumption that $P \neq NP$) there is no known algorithm that is able to find an optimal solution in polynomial time for every instance of the BPP. Instead, heursitics can be used to find a near-optimal solution in a shorter amount of time. One example is the greedy heuristic known as first-fit, an online algorithm that places each item (in some arbitrary order) into the lowest-indexed bin. If there are no bins that can feasibly accommodate the item, the algorithm places the item into a new bin. A modified version of first-fit yields the well-known first-fit decreasing (FFD) algorithm (\citealp{eilon1971}), which initially sorts the items into non-decreasing order of size before allocating them into bins. As proven by \cite{dosa2007} (tight bound), the worst case for FFD is $\frac{11}{9}k + \frac{6}{9}$ bins, where $k$ is the optimal number of bins given by $\ceil{\frac{\sum_{i=1}^{n} w_{i}}{c}}$ (..if the solution has this many bins then it is optimal, (\citealp{korf2002})).

In 2004, Goulimis brought to light an open-combinatorial problem related to the bin-packing problem. Corrugated boxes are formed in two stages: firstly, they are produced flat, then a device consistsing of knives mounted on a bar creates ``scores'' along specific lines on the flat boxes. These scores allow the boxes to be folded in predetermined places. The first stage of this process involves finding a feasible pattern of the flat boxes that minimises the amount of waste, and has been researched extensively. It is classically solved by delayed column-generation (\citealp{gilmore1961, gilmore1963}). However, the second step of this process requires an additional constraint.

The scoring knives, by design, cannot be placed too close together (around 70mm in industry), and as such have a ``minimum knife distance''. The problem then becomes a bin-packing problem with an additional constraint: find a feasible arrangement of boxes such that the sum of two adjacent score widths from different boxes is greater than or equal to the minimum knife distance. We will call this constraint the minimum total score width constraint.

\textcolor{red}{INSERT FIGURE TWO BOXES ANNOTATED AND KNIVES ON BAR FOLD LINES SCORE WIDTH}

\textcolor{blue}{Figure showing two boxes with score widths and the scoring knives mounted on a bar. Since the sum of the two adjacent score widths from different boxes is greater than or equal to 70mm, the minimum total width constraint has been met - this is a feasible alignment of these two boxes}

There exists a polynomial time algorithm that is able to recognise whether a particular instance of the problem is feasible or infeasible (\citealp{becker2010}) (i.e. whether all boxes can be arranged in one line). However, the algorithm does not consider the lengths of each individual box, or the total length of the final arrangement of boxes. In industry, strips of materials are provided in fixed lengths, so, given a large number of boxes to pack, it may not be feasible to pack all of the boxes onto one strip. 

This leads us to our main problem, the score-constrained bin-packing problem: given a collection of boxes with varying widths and score widths on either side, and strips of a fixed length, find the minimum number of strips required to accommodate all of the boxes, such that the minimum total width constraint is met between all boxes on all strips. 

The remainder of this report will firstly, in Section \ref{sec:ffdapprox}, introduce the first-fit decreasing approximate algorithm and describe the modifications made to consider the minimum score width constraint. In Section \ref{sec:ffdexact}, we will present a new algorithm which consists of the modified first-fit decreasing algorithm detailed in the previous section combined with a polynomial-time algorithm which will be used to find feasible sub-solutions. A comparison of the two algorithms and an analysis of the results will be provided in Section \ref{sec:compresult}, and finally Section \ref{sec:conclusion} concludes the paper and proposes some potential directions for furture work.

\begin{itemize}
	\item FF if online, FFD is offline
	\item FF and FFD can produce a feasible solution quickly, but the solution may not be optimal O(nln n), where n is the number of items that need to be packed
	\item Exact algorithms, Martello and Toth (MTP), Bin Completion algorithm Korf 2002, Schreiber and Korf (2013) (wiki)
	\item \textcolor{blue}{Although there are exact solutions available for the BPP} (martello knapsack MTP 1989 (1990b) ``is based on a first-fit decreasing'' branching strategy and is the best existing algorithm for optimal packing (KOrf2002)'', hung and brown 1978 ``a branch-and-bound algorithm'' paper title:``an algorithm for a class of loading problems'', eilon christofides 1971 ``a depth-first numerative algorithm'' paper title: ``the loadings problem'', schreiber and korf 2002 bin completion algorithm) \textcolor{blue}{they are only able to find solutions to problems that have a small input size.}
	\item ``Hence, exact solution techniques are bound to work well for small to medium sized problem instances only, and real world sized problems including up to thousands of items have to be solved heuristically.''
	\item ``There exists at least one ordering of items that allows FF to produce an optimal solution'' (wiki) Rhyd Lewis 2009
	\item Picture of box with score widths, score lines
	\item All boxes have same height, different widths, different score widths
	\item Recognition algorithm Becker
	\item Formal statement of BPP with score constraint called the Score Constrained Bin packing problem (SCBPP): given
	\item dicuss other paper/articles with this problem, i.e. rhyd 2017	
\end{itemize}

\section{Formulating the problem}
We are given $n$ boxes that need to be packed onto strips. Each of these boxes have a positive width, and two score widths, one on either side of the box. Since these score widths are not necessarily equal, the boxes can be aligned in two ways: either in a ``regular'' alignment, where the smaller of the two score widths is on the left hand side of the box, or in a ``rotated'' alignment, where the box has been rotated by $180^{\circ}$ so that the larger score width is on the left-hand side of the box. Strips are packed left to right. We also have a finite number of strips of equal fixed lengths. Once a box has been placed onto a strip, the rightmost score on the strip is referred to as the ``free'' score. A box can only be placed onto a strip if it can fit onto the strip, i.e. the addition of the box to the strip doesn't exceed the length of the strip, and if the free score and one of the scores on the box meet the minimum total score width constraint, i.e. the sum of the free score width and one of the scores on the box is greater than or equal to the minimum total score width. Assume that no box has a width that is greater than the fixed length of the strips.

Definition: Score-Constrained Bin-Packing Problem. Find the minimum number of strips required to feasibly pack all the boxes such that the minimum total score width constraint is met between every pair of adjacent score widths from different boxes on every strip. Then the SCBPP consists of finding a feasible alignment of all of the boxes onto the strips such that every pair of adjacent score widths from different boxes meets the minimum total score width constraint, and that the least number of strips is used.


\section{FFD Using Approximate Algorithm}
\label{sec:ffdapprox}
\begin{itemize}
	\item discuss heuristics
	\item ``Heuristics are approximate solution techniques that typically provide solutions in the least amount of time to the detriment of the solutions quality''
	\item ``Exact methods find a best possible packing, but are slow and may be unable to provide solutions to large or reastically sized problem instances within reasonable time''
	\item ``Consider a set of $n$ items. In the context of bin packing problems, heuristics consider these items one-by-one and attempt to pack them into a bin so as to achieve a near-optimal solution without the guarantee that a solution is optimal.''
	\item ``Exact packing methods are often called deterministic methods and are guarantees to find an optimal solution to a problem (given sufficient time and a feasible region) while heuristics attempt to find optimal solutions, but are not guaranteed to do so.''
	\item ``Exact methods may often be too slow to solve large problem instances, but may typically be used to solve smaller ones.''
	\item ``Packing heuristics follow a fixed set of rules to pack items in such a manner as to find good, feasible (but not necessarily optimal) solutions to the bin packing problem within as short a time span as possible.''
	\item offline algorithm, whole problem data is provided/required at the start.
	\item online, doesn't consider future events/items, data is processed in the order that it is input into the algorithm, entire input is not required to be available from the start
	\item ``first-fit decreasing (FFD) algorithm for one-dimensional bin packing by johnson et al (1974, worst-case performance bounds for simple one-dimensional packing algorithms)''
	\item FF is online, FFD is offline
	\item FF and FFD can produce a feasible solution quickly, but the solution may not be optimal, O(nln n), where n is the number of items that need to be packed
	\item who stated that ffd is O(nlog n)?
	\item Coffman et al 1984 heuristics survey
	\item \textcolor{blue}{Describe how the ffdapprox algorithm works}
	\item \textcolor{red}{figure of packed boxes on strips}
	\item previously placed boxes are fixed, cannot be rearranged
	\item complexity of modified FFD for SCBPP
	\item background information on FFD
	\item works in the same way as FFD, but with an additional step - each box must meet the minimum total score width constraint, so if a box fits onto a strip, we start by checking if the smallest of the two score widths on the box, and the free score width on the end of the current strip, meet the mtswc. If not, then we rotate the box $180^{\circ}$ and check again if the mtswc is met. If neither score width on the box meets the mtswc with the free score width, then we move onto the next strip and try again.
	\item pack from left to right
	\item boxes can be aligned in two ways, with smaller score on LHS, or rotated $180^{\circ}$ so the smaller score is on the LHS
	\item Karp complexity
	\item \textcolor{red}{figure of one packed strip indicating free score end $e$}
	\item pros and cons: speed, however there may be a solution that exists that cannot be found due to boxes being fixed in place
	\item In this algorithm, the boxes are sorted by non-increasing widths prior to packing.
	\item If a box does not fit onto any existing strips, a new strip is initialised.
	\item Initially, pack the largest (widest/box with the largest width/first box in the sorted list of boxes) box into the first strip.
	\item ``Simchi-Levi (1994, ``New worst-case results for the bpp'') has shown that the FFD and the BFD heuristics have an absolute performance ratio of 1.5, and that this value is the best possible for the BPP, unles $P = NP$.''
	\item We use $\{i_1, i_2\}$ or $\{i_2, i_1\}$ to represent the box, rather than just $i$, because the feasibility of the alignment of boxes on the strip depends on which way the box is placed, regularly or rotated.
\end{itemize}

\begin{algorithm}[H]
	\caption{\textcolor{red}{First-Fit Decreasing Algorithm for the Score-Constrained Bin-Packing Problem}}
	\label{alg:ffdapprox}
	\begin{algorithmic}[1]
	\Require $n$ boxes with widths $w(i)$ for $i = 1,2,...,n$, each box $i$ has two scores $i_1$ (smaller score width) and $i_2$ (larger score width), where $w(i_1) \leq w(i_2)$. Let $\{i_1, i_2\}$ represent box $i$ aligned such that the smaller score is on the LHS, and let $\{i_2, i_2\}$ represent box $i$ rotated by $180^{\circ}$ such that the larger score is on the LHS. Strips are packed from left to right. Finite number of strips $S_j, j = 1, 2, ...$ that all have equal fixed length $l > 0$, and a residual length $r(j)$, where initially $r(j) = l \forall j = 1, 2, ...$. Initially $S_j = \emptyset \forall j = 1, 2,...$. Let $w(f_j)$ denote the width of the free score for strip $j$.   
	\State Sort boxes in non-increasing order of widths, breaking ties arbitrarily
	\State $i \gets 1, j \gets 1$
	\State $S_j \gets S_j \cup \{i_1, i_2\}$
	\State $r(j) \gets r(j) - w(i)$
	\For{$i \gets 2$ \textbf{to} $n$}
		\For{$j \gets 1$ \textbf{to} $n$}
			\If{$S_j \neq \emptyset \land w(i) \leq r(j)$}
				\If{$w(f_j) + w(i_1) \geq t$}
					\State $S_j \gets S_j \cup \{i_1, i_2\}$
					\State $r(j) \gets r(j) - w(i)$
					\State \Break
				\ElsIf{$w(f_j) + w(i_2) \geq t$}
					\State $S_j \gets S_j \cup \{i_2, i_1\}$
					\State $r(j) \gets r(j) - w(i)$
					\State \Break					
				\EndIf
			\ElsIf{$S_j = \emptyset$}
				\State $S_j \gets S_j \cup \{i_1, i_2\}$
				\State $r(j) \gets r(j) - w(i)$
				\State \Break
			\EndIf
		\EndFor
	\EndFor		
	\Ensure all boxes packed onto strips while meeting minimum total width constraint. Set $S$ is all strips with boxes in correct rotation to meet minimum total score width constraint.	
	\end{algorithmic}	
\end{algorithm}

DESCRIPTION OF FFDAPPROX 
\begin{itemize}
	\item start by sorting all boxes into non-increasing order of widths, breaking ties arbitrarily
	\item starting with the first box in the list (i.e with the box that has the largest width), put it onto the first strip, with the box rotated such that the side of the box with the smaller score width is placed at the beginning of the strip
	\item for each box, find the lowest indexed strip that the box can fit into
	\item if the box can fit onto a strip, check if the free score width on that strip and the smaller of the two score widths on the box meet the minimum total score width requirement. If so, place the box onto that strip such that the smaller of the two score widths is placed next to the free score width. Move onto the next box in the list
	\item else, check if the larger of the two score widths on the box and the free score width meet the minimum total score width constraint. If so, place the box onto that strip such that the larger of the two score widths is placed next to the free score width. Move onto the next box.
	\item else, if neither of the two score widths and the free score meet the minimum total score width constraint, move onto the next used strip and repeat
	\item if there are no used strips that can accommodate the current box, place the box into a new unused strip, with the smaller score at the beginning of the strip (on the LHS). move onto the next box.
	\item we then obtain ...
\end{itemize}



\section{FFD Using Exact Algorithm}
\label{sec:ffdexact}
Introduction
\begin{itemize}
	\item This alg will be combining a well known heuristic with an exact algorithm to find a feasible solution to the SCBPP
	\item Becker 2010 recognicity algorithm in polynomial time
	\item doesn't take into account individual widths of boxes
	\item only one strip/align all boxes into one row
	\item in industry, strips have a finite length
	\item algorithm also doesn't output a full solution, only whether there exists a feasible solution or not
	\item ffd but including an algorithm that can find a feasible aligment of boxes with score constraints if one exists
	\item unlike ffd, where boxes are placed onto the end of each strip, the boxes are not fixed in place, all boxes can be reordered.
	\item Can find a solution for a set of boxes on a strip when it may not have been possible to find one in ffdapprox due to only the last box being checked.
	\item try to find a hamiltonian path 
	\item complexity $O(n^2), O(n)$ to construct a solution
	\item if the algorithm cannot find a feasible alignment of the boxes including the current box, move onto next strip and try again.
\end{itemize}

Explanation: Initialise

Let each box be represented by two vertices on a graph $G$, one vertex for each side of the box. We also introduce a ``dummy'' box that has two ``dominating scores'' with widths equal to the minimum total score width, $t$. This dummy box will be discarded at the end of the algorithm, and will not be included in the final feasible alignment. We now have $n$ boxes including the dummy box, and so $G$ will have $2n$ vertices.

We will refer two two vertices that represent either side of the same box as ``partners''. The boxes cannot be altered in any way, and so partners must be adjacent to one another. We therefore create a set of edges, $P$, consisting of edges between partners on $G$. \textcolor{red}{weighted edges $P$}.

Each vertex $v_i$ for $i = 1, 2, ..., 2n$ has a positive value $w(v)$ corresponding to the score width it represents. There is an edge between two vertices $u$ and $v$ on $G$ if and only if the vertices fulfil the \textit{minimum total score width constraint}, 

\begin{equation*}
w(u) + w(v) \geq t,
\end{equation*}

and $u$ and $v$ are not partners, i.e. $\{u, v\} \notin P$. The vertices $u$ and $v$ are referred to as \textcolor{red}{``matched''}, and so their corresponding score widths can be feasibly aligned next to one another. We now have another set of edges, $Q$, consisting of edges between \textcolor{red}{``matched''} vertices.  

In order to find a feasible alignment of the boxes, we must find an alternating Hamiltonian cycle on $G$. \textcolor{red}{turns into path when we remove dummy box.}

\begin{definition}
	\label{defn:althamcycle}
	Let $G(V, P\cup Q)$ be an undirected graph, where each edge is a member of one of two sets, $P$ or $Q$. $G$ has an \textit{alternating Hamiltonian cycle} if there exists a Hamiltonian cycle on $G$ such that the successive edges alternate between sets $P$ and $Q$. For example, if vertex $v_k$ in the Hamiltonian cycle is adjacent to vertex $v_{k-1}$ via an edge from $P$, then $v_k$ must be adjacent to vertex $v_{k+1}$ via an edge from $Q$, or vice versa. 
\end{definition}

\textcolor{red}{Figure of graph $G$ with boxes, figure of aligned boxes with alternating Hamiltonian path underneath.}

\textcolor{red}{we now need to find a way of producing the alternating hamiltonian cycle on the graph $G$, if one exists.}

MTGMA/MIS

We need to obtain a set of matching edges $M \subset Q$ of cardinality $n$, such that the matching edges from $M$ and the edges from $P$ form an alternating Hamiltonain cycle as described in Definition \ref{defn:althamcycle}. We use a modified matching algorithm (\citealp{mahadev1994, becker2010}) that produces a suitable set of matching edges $M$. 

Initially, the vertices are sorted in order of non-decreasing values, and ties are broken arbitrarily. Starting from the smallest vertex, find the largest adjacent vertex (under $Q$? in the set $Q$? )that is not the current vertex's partner, and that has not yet been matched. Add this pair of vertices to the set $M$ that keeps a record of the matching edges, and proceed to the next vertex in the list. Continue until all vertices have been assessed. 

If a vertex $v_i$ is not adjacent to any other vertex except its partner, we can rematch previous vertices to provide $v_i$ with a feasible match, provided $v_i$ is not the first vertex in the sorted list, the previous vertex $v_{i-1}$ has a match (i.e. $v_{i-1} \in M$), and $v_{i-1}$ is adajcent to $v_i$'s partner. We then simply match $v_{i-1}$ with $v_i$'s partner, and match $v_i$ with $v_{i-1}$'s previous match. \textcolor{red}{max cardinality.}

If $|M| < n$, then there are not enough matching edges to form an alternating Hamiltonian cycle with the edges in $P$, and therefore there is no feasible arrangement of the boxes such that the minimum total score width constraint is not violated. However, if $|M| = n$, then the graph $G(V, P \cup M)$ is a 2-regular graph, where each vertex is adjacent to one vertex via an edge in $P$, (i.e. its partner), and another vertex via an edge in $M$ (i.e. its match). If $G$ consists of one alternating Hamiltonian cycle, then we simply remove the dominating vertices to obtain a feasible alignment of the boxes/feasible solution. If there are more than two cycles on $G$, we need to combine them into one alternating Hamiltonian cycle.


FCA/Patch
\begin{itemize}
	\item Find pairs of vertices in each cycle connected via a matching edge/ edge from $M$ that can be broken and rematched.
	\item Break/remove matching edges between vertices and create/add new matching edges such that the cycles are combined/connected.
	\item If there are no vertices that can be rematched from different cycles, then the cycles cannot be joined together, and therefore there is no feasible solution.
	\item Else, if we are able to join the cycles together to create one alternating Hamiltonian cycle, we then simply have to remove the dominating vertices to obtain a final solution/feasible alignment of the boxes.
	\item Sort edges, successive edges, see if edges are from different cycles and if the smaller vertex of the lower indexed edge is adjacent to the larger vertex of the higher indexed edge.
\end{itemize}


\begin{algorithm}[H]
	\caption{\textcolor{blue}{First-Fit Decreasing Algorithm for the Score-Constrained Bin-Packing Problem}}
	\label{alg:ffdexact}
	\begin{algorithmic}[1]
	\Require fill in.   
	\State Sort boxes in non-increasing order of widths, breaking ties arbitrarily
	\State $i \gets 1, j \gets 1$
	\State $S_j \gets S_j \cup \{i_1, i_2\}$
	\State $r(j) \gets r(j) - w(i)$
	\For{$i \gets 2$ \textbf{to} $n$}
		\For{$j \gets 1$ \textbf{to} $n$}
			\If{$S_j \neq \emptyset \land w(i) \leq r(j)$}
				\State Perform \textcolor{red}{algorithm} on boxes in $S_j$ and box $i$
				\If{feasible alignment of boxes found}
					\State Replace alignment of boxes in $S_j$ with new box alignment including box $i$
					\State $r(j) \gets r(j) - w(i)$
					\State \Break
				\EndIf
			\ElsIf{$S_j = \emptyset$}
				\State $S_j \gets S_j \cup \{i_1, i_2\}$
				\State $r(j) \gets r(j) - w(i)$
				\State \Break
			\EndIf
		\EndFor
	\EndFor		
	\Ensure all boxes packed onto strips while meeting minimum total width constraint. Set $S$ is all strips with boxes in correct rotation to meet minimum total score width constraint.	
	\end{algorithmic}	
\end{algorithm}

DESCRIPTION FFD EXACT
\begin{itemize}
	\item Start with boxes sorted in order of non-increasing widths, breaking ties arbitrarily.
	\item Place the first box onto the first strip, such that the smaller of the two score widths is on the leftmost side of the strip.
	\item Attempt to fit each box $i, i = 2,..., n$ onto the lowest-indexed strip possible that already contains a box, i.e. the strip is not empty.
	\item If a box is able to feasibly fit onto a strip, apply the \textcolor{red}{algorithm} on all of the boxes on the strip plus the current box $i$.
	\item If a feasible alignment of the boxes has been found, replace the current alignment of boxes on the strip with the new alignment, which includes box $i$ (and move onto the next box).
	\item If no feasible alignment of the boxes is found, we move onto the next lowest-indexed strip and repeat.
	\item If a box cannot fit into any of the used strips, i.e. any of the strips that already contain a box/any of the open strips, we then place the box onto a new, unused strip, such that the smaller of the two score widths on the box is on the leftmost side of the strip, then move onto the next box.
	\item Continue untill all boxes have been placed onto a strip/until all boxes have been packed.
\end{itemize}





\section{Comparison and Results}
\label{sec:compresult}

\begin{itemize}
	\item ``Dowsland and Dowsland (1992, ``Packing Problems'') comment that while average or worst-case performance bounds are useful guidelines, it is best to determine an algorithm's usefuless by testing it on data sets typical to the intended problem. Repositories are available on the internet where benchamrk data sets for packing problems are stored for the purpose of evaluating algorithms.''
\end{itemize}

\section{Conclusion}
\label{sec:conclusion}

\section{References}
\cite{becker2010}: Twin Constrainted Hamiltonain Paths on Threshold Graphs 

\cite{becker2015}: A Heuristics for the MSSP

\cite{coffman1978}: An Application of Bin-Packing to Multiprocessor Scheduling

\cite{coffman1984}: Approximation Algorithms for Bin-Packing - An Updated Survey.
\begin{itemize}
	\item ``first fit and first fit decreasing algorithms have equivalent worst-case time complexities of O(n ln n) (cite coffman1984)''
	\item ``we refer the interested reader to the excellent survery by (cite coffman1984), which includes a bibliography of more than one hundred titles''
	\item ``A comprehensive review of various heuristic algorithms is provided in a recent survery by (cite coffman1984)''
	\item ``(cite coffman1984) did an excellent survey on this problem, particularly on approximation algorithms and their asymptotic performance ratios''
\end{itemize}
\cite{dosa2007}: The tight bound of first fit decreasing bin packing algorithm

\cite{eilon1971}: The loading problem. 
\begin{itemize}
	\item ``(cite eilon1971) presented a depth-first enumerative algorithm''
	\item ``(cite eilon1971) discuss its(the BPP) application to the loading of vehicles (or other containers) with consignment'' 
	\item ``the most well-known heuristics are the FFD and the BFD (cite eilon1971)''
\end{itemize}
\cite{garey1979}: Computers and Intractibility - A Guide to the Theory of NP-Completeness. 
\begin{itemize}
	\item ``(cite garey1979) cite simple heuristics which can be shown to be no
	worse(but also no better) than a rather small multiplying factor above the optimal number of bins.''
	\item ``the bpp problem is NP-Hard''
\end{itemize}

\cite{gilmore1961}: A LP Approach to the CSP Part 1

\cite{gilmore1963}: A LP Approach to the CSP Part 2

\cite{goulimis2004}: Minimum Score Separation

\cite{johnson1974}: Worst case performance bounds for simple one-dimensional packing algorithms
\begin{itemize}
	\item ``(cite johnson1974) discuss three practical applications in computer science, which are table formatting, prepaging, and file allocation''
\end{itemize}

\cite{karp1972}: Reducibility among combinatorial problems

\cite{korf2002}: A new algorithm for optimal bin packing. ``If the number of bins in a solution is equivalent to the expression (sum of all weights/ capacity) then that solution is an optimal solution to the problem instance (cite Korf2002)'' 

\cite{lewis2009}: A general-purpose hill-climbing method for order independent minimum grouping problems

\cite{lewis2017}: How to pack trapezoids

\cite{lewis2011}: An investigation into two bpp with ordering and orientation implications

\cite{mahadev1994}: Longest cycle in threshold graphs

\cite{mahadev1995}: Threshold graphs and related topics

\cite{martello1990b}: Knapsack problems. ``(cite martello1990) proposed a branch-and-bound based exact algorithm MTP for the BPP.'' 






\bibliographystyle{dcu}
\bibliography{includes/bibliography}

%\input{includes/bibliography.tex}


%%Additional Comments
\begin{comment}
- all citations have been changed to \cite{}, so that the citation shows up in green in latex. some of these citations need to be changed to (\citealp{}) (or \citep{}, same outcome) as the citations need to be in brackets but without the year in brackets.
- 1. "As proven by \cite{dosa2007}, ..." -> "As proven by Dosa (2007), ..."
- 2. "... FFD algorithm (\cite{eilon1971}), which ..." -> "... FFD algorithm (Eilon and Christofides (1971)), which .." NOTE THE YEAR IS IN BRACKETS INSIDE THE CITATION BRACKETS, THIS IS ONLY WHILE WRITING, THIS NEEDS TO BE CHANGED BEFORE SUBMISSION
- 3. "... FFD algorithm (\citealp{eilon1971}), which ..." -> "... FFD algorithm (Eilon and Christofides, 1971), which .." THIS IS HOW IT NEEDS TO BE WHEN SUBMITTING
\end{comment}
















\end{document}

