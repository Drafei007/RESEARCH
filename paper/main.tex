\documentclass[oribibl]{llncs}

%---- PREAMBLE ----%

% Allows separate files to be used in the main file
\usepackage{subfiles}
\usepackage{natbib}
\bibpunct{(}{)}{;}{a}{,}{,}

% Algorithms
\usepackage{algorithm}
\usepackage{algpseudocode}
\renewcommand{\algorithmicrequire}{\textbf{Input:}}
\renewcommand{\algorithmicensure}{\textbf{Output:}}
\newcommand{\IndStatex}[1][1]{\Statex\hspace{6mm}}
\newcommand{\IndIndStatex}[1][1]{\Statex\hspace{12mm}}
\newcommand{\IndIndIndStatex}[1][1]{\Statex\hspace{18mm}}
\newcommand{\IndIndIndIndStatex}[1][1]{\Statex\hspace{24mm}}
\newcommand{\IndIndIndIndIndStatex}[1][1]{\Statex\hspace{30mm}}
\newcommand{\IndIndIndIndIndIndStatex}[1][1]{\Statex\hspace{36mm}}

% Contains advanced math extensions
\usepackage{amsmath}

% Introduces the *proof* environment and the \theoremstyle command
%\usepackage{amsthm}

% Adds new symbols to be used in math mode, e.g. \mathbb
\usepackage{amssymb}

% To declare multiple authors
\usepackage{authblk}

% Provides extra comands as well as optimisation for producing tables
\usepackage{booktabs}

% Allows customisation of appearance and placements for figures/tables etc.
\usepackage{caption}

% Adds support for arbitrarily-deep nested lists
\usepackage{enumitem}

% Improves the interface for defining floating objects such as figures/tables
\usepackage{float}

\usepackage{fullpage}

% For easy management of document margins and the document page size
%\usepackage{geometry}

% Allows insertion of graphic files within a document
\usepackage{graphicx}

% Manage links within the document or to any URL when you compile in PDF
\usepackage{hyperref} 
\usepackage{xcolor}
\hypersetup{
	colorlinks,
	linkcolor={red!50!black},
	citecolor={green!30!black},
	urlcolor={blue!80!black}
}

% Successor of amsmath
\usepackage{mathtools}

% No indentation, space between paragraphs
\usepackage{parskip}

%Include standalone .tex files
\usepackage{standalone}

% Define multiple floats (figures/tables) within one environment with individual captions 1a, 1b etc
\usepackage{subcaption}

\usepackage{tikz}

\usepackage{wrapfig}

\DeclarePairedDelimiter{\floor}{\lfloor}{\rfloor}
\DeclarePairedDelimiter{\ceil}{\lceil}{\rceil}

%Theorem style
%\theoremstyle{plain}% default
%\newtheorem{theorem}{Theorem}[section]
%\newtheorem{corollary}{Corollary}[theorem]

%\theoremstyle{definition}
%\newtheorem{defn}{Definition}[section]
%\newtheorem{proposition}{Proposition}[defn]
%\newtheorem{exmp}{Example}[section]

\title{Heuristics for the Score-Constrained Strip-Packing Problem}
\author{Asyl L. Hawa \and Rhyd M. R. Lewis \and Jonathan M. Thompson}
\institute{School of Mathematics, Cardiff University, Senghennydd Road, Cardiff, UK, CF24 4AG}
\date{\today}

\begin{document}
\maketitle

\begin{abstract}
	
\end{abstract}

\section{Introduction}
\label{sec:intro}

Firstly, let us introduce the Constrained Pair-Ordering Problem:

\begin{definition}
	\label{defn:cpop}
	Let $\mathcal{M}$ be a multiset of unordered pairs of positive integers $\mathcal{M} = \{\{a_1, b_1\}, \{a_2,b_2\},...,\{a_n,b_n\}\}$, and let $\mathcal{T}$ be an ordering of the elements of $\mathcal{M}$ such that each element is a tuple. The Constrained Pair Ordering Problem (CPOP) consists of finding a solution $\mathcal{T}$ such that, given a fixed value $\tau \in \mathbb{Z}^{+},$
	\begin{equation}
		\label{eqn:cpop}
		\textup{\textbf{rhs}}(i) + \textup{\textbf{lhs}}(i+1) \geq \tau \hspace{5mm} \forall \hspace{1mm} i \in \{1,2,..., n-1\},
	\end{equation}
	where \textup{\textbf{lhs}($i$)} and \textup{\textbf{rhs}($i$)} denote the left- and right-hand values of the $i$th tuple. The inequality is referred to as the vicinal sum constraint.
\end{definition}

For example, given the multiset $\mathcal{M} = \{\{1,2\}, \{1,5\}, \{2,4\}, \{3,4\}, \{4,5\}\}$ and $\tau = 7$, one possible feasible solution is $\mathcal{T} = \langle(1,2), (5,4), (3,4), (4,2), (5,1) \rangle$.

One prominent application of the CPOP is in a strip-packing problem brought to light as an open-combinatorial problem by \citeauthor{goulimis2004} in 2004. A set $\mathcal{I}$ of rectangular items of equal height $h$ made from cardboard are to be packed onto a strip from left to right. Each item $i \in \mathcal{I}$ has width $w_i \in \mathbb{Z}^{+}$, and possesses two vertical score lines, marked in predetermined places. A pair of knives mounted onto a bar cuts along the score lines of two adjacent items simultaneously, which allows the items to be folded with ease (see Figure \ref{fig:boxknife}). However, by design, the scoring knives cannot be placed too close to one another, and as such have a ``minimum scoring distance'' (around 70mm in industry). The distances between each score line and the nearest edge on an item $i \in \mathcal{I}$ are the score widths, $a_i, b_i \in \mathbb{Z}^{+}$, assigned such that $a_i \leq b_i$. Since these score widths are not necessarily equal, an item $i$ can be positioned in one of two orientations: ``regular'', denoted $(a_i, b_i)$, or ``rotated'', denoted $(b_i, a_i)$, where the smaller of the two score widths $a_i$ is on the left- and right- hand side of the item respectively. Clearly, for two items to be placed alongside one another feasibly the sum of the adjacent score widths must equal or exceed the minimum scoring distance, else the knives will not be able to score the items in the correct locations. Thus, the problem consists of finding a suitable ordering and orientation of the items such that the sum of every pair of adjacent score widths is greater than or equal to the minimum scoring distance. Specifically, as the items are packed from left to right, this involved checking the sum of the right-hand score width of each item $i$ and the left-hand score width of the next item $i+1$ on the strip. The left-hand score width of the first item and the right-hand score width of the last item on the strip are not adjacent to any other item, and are therefore ignored. 


\begin{figure}[h!]	
	\centering
	\includestandalone[width=0.8\textwidth]{figures/boxesknifeannotate}
	\caption{\textcolor{red}{Individual boxes.}}	
	\label{fig:boxknife}
\end{figure}

It can be seen that this problem is in fact analogous to the CPOP, where each unordered pair in an instance $\mathcal{M}$ contains values corresponding to the score widths of an item, and $\tau$ is the minimum scoring distance. Thus, a set $\mathcal{I}$ of items has a feasible arrangement if the equivalent CPOP has a solution $\mathcal{T}$.

Observe that, in this particular strip-packing problem, the widths of the individual items are not taken into account, since the aim is to arrage the items onto a strip of seemingly infinite length. In reality, strips of material are provided in fixed lengths. Given a large problem instance, multiple strips may be required to feasibly accommodate all of the items. For this reason, we define a new problem to be investigated.

\begin{definition}
	\label{defn:scspp}
	Given a set $\mathcal{I}$ of rectangular items of varying widths $w_i$ and score widths $a_i, b_i$ for each item $i \in \mathcal{I}$, a finite set of strips $S_j$ with equal fixed length $l$, and a minimum scoring distance $\tau$, the Score-Constrained Strip-Packing Problem (SCSPP) consists of finding the minimum number of strips required to pack all items in $\mathcal{I}$ such that the minimum total score width constraint is met between every pair of adjacent items on every strip.
\end{definition}

The remainder of this article will firstly, in Section 2, detail a polynomial-time algorithm that is able to produce a solution to any instance of the CPOP, if a solution exists. Section 3 describes the heuristics that will be used to find feasible solutions to the SCSPP, and discusses the associated advantages and disadvantages. A comparison of these heuristics and an analysis of the results will be provided in Section 4, and finally Section 5 concludes the paper and proposes some potential directions for future work.


\section{Solving the CPOP}
\label{sec:ahca}

We shall now present an algorithm for solving the CPOP, which operates by expressing the problem instance graphically. 

Let $\mathcal{M}$ be an instance of the CPOP of cardinality $n$ as described in Definition \ref{defn:cpop}. Then, the graph $G$ is defined by a vertex set $V = \{v_1, v_2, ...v_{2n+2}\}$, where each vertex is assigned a value from $\mathcal{M}$ such that $v_1 \leq v_2 \leq ... \leq v_{2n+2}$. Note that there are two extra ``dominating'' vertices, $v_{2n+1}$ and $v_{2n+2}$, which have values equivalent to $\tau$. In $\mathcal{M}$, the values that make up a pair are referred to as ``partners''. Thus, let $p: V \to V$ be a bijective function that associates each vertex $v_i \in V$ on $G$ with its partner, $v_j = p(v_i)$. $G$ comprises two distinct edge sets: $P$, which contains edges between vertices that are partners, and $E$, which consists of edges between vertices whose corresponding values in $\mathcal{M}$ can be arranged alongside one another. That is, for two distinct vertices $v_i, v_j$,

\begin{equation}
	\label{eqn:jointsum}
	v_i + v_j \geq \tau,
\end{equation}

provided $p(v_i) \neq v_j$. Hence, $G(V, P \cup E)$ is an undirected graph with $2n + 2$ vertices and edge sets $P$ and $E$, such that $P \cap E = \emptyset$. Observe that $P$ is in fact a perfect matching, with $|P|= n+1$.

The following definition describes a variant of the classical Hamiltonian cycle on a graph $G$.

\begin{definition}
	\label{defn:althamcycle}
	Let $G(V, P\cup E)$ be an undirected graph, where each edge is a member of one of two sets, $P$ or $E$. $G$ contains an alternating Hamiltonian cycle if there exists a Hamiltonian cycle such successive edges alternate between sets $P$ and $E$. For example, if $(v_{i-1}, v_i) \in P$, then $(v_i, v_{i+1}) \in E$, or vice versa.
\end{definition}

\begin{theorem}
	\label{thm:cpopsoln}
	There exists a feasible solution $\mathcal{T}$ to an instance $\mathcal{M}$ of the CPOP if and only if its corresponding graph $G(V, P\cup E)$ contains an alternating Hamiltonian cycle.
\end{theorem}
\begin{proof}
	To be completed. \qed
\end{proof}

Therefore, the aim of this algorithm is to construct an alternating Hamiltonian cycle on $G$. Since the values in each unordered pair in $\mathcal{M}$ cannot be altered, every edge in $P$ must be present in the alternating Hamiltonian cycle. Consequently, the task now involves finding a matching $Q \subset E$ of cardinality $n+1$, such that the edge sets $P$ and $Q$ form an alternating Hamiltonian cycle as described in Definition $\ref{defn:althamcycle}$.  


To find a matching $Q \subset E$, the following modified maximum cardinality matching algorithm is executed (\citealp{mahadev1994, becker2010}): for each vertex $v_i$, $i = 1, 2,...,2n+2$, find the largest vertex $v_j \in \Gamma(v_i)\backslash p(v_i)$ (i.e. the largest vertex $v_j$ in $v_i$'s neighbourhood that is not $v_i$'s partner) that is not incident to any edge in $Q$. Add the edge connecting these two vertices to $Q$, and proceed to the next vertex until all vertices have been assessed. The pair of vertices incident to each edge in $Q$ are referred to as being ``matched'', and are connected by a ``matching'' edge. 

During this algorithm, if a vertex $v_i$ is not adjacent to any other unmatched vertex except its partner $p(v_i)$, the preceeding vertex $v_{i-1}$ can be ``rematched'', provided $v_i$ is not the first vertex (i.e $i \neq 1$), $v_{i-1}$ has been matched, and $(v_{i-1}, p(v_i)) \in E$. Then, $v_{i-1}$ is matched with $p(v_i)$, and $v_i$ is matched with the vertex that was previously matched with $v_{i-1}$. It is guaranteed that this algorithm will produce a maximum cardinality matching $Q \subset E$. This algorithm is able to create a matching set $Q$ in at most $O(n \log n)$ time.

Clearly, if $|Q| < n+1$, there are not enough matching edges to form an alternating Hamiltonian cycle along with the edges in $P$, and hence in accordance with Theorem $\ref{thm:cpopsoln}$ no feasible solution exists for the given instance $\mathcal{M}$ of the CPOP. Otherwise, if $Q$ is a perfect matching of cardinality $n+1$, the spanning subgraph $G'(V, P \cup Q)$ is a 2-regular graph, where each vertex is adjacent to its partner via an edge in $P$, and its match via an edge in $Q$. $G'$ thus consists of a collection of alternating cycles $A_1, A_2, ..., A_{\lambda}$. If there is only one component on $G'$, i.e. $\lambda = 1$, then this alternating cycle is in fact Hamiltonian, and therefore a feasible solution exists. For a graph $G'$ with $\lambda > 1$, the components of $G'$ must be combined to form one single alternating Hamiltonian cycle. The algorithm thus executes a procedure that selects suitable edges in $E$ to replace specific edges in $Q$, such that the components of $G'$ are merged together.

For each edge $(v_i, v_j) \in Q$, where $i < j$, $v_i$ is referred to as the ``smaller'' vertex, and $v_j$ as the ``larger'' vertex, since the value associated with $v_i$ is less than or equal to the value associated with $v_j$. The procedure initially sorts the edges in $Q$ into a list such that the smaller vertices of the edges are in lexicographical order, and therefore in non-decreasing order of values. Any edges that are a result of rematching are then removed from this list. Starting from the first edge, the procedure searches through the list to find an edge that meets the following conditions:
\begin{enumerate}
	\item The smaller vertex of the current edge and the larger vertex of the succeeding edge are connected by an edge in $E$.
	\item The current edge and the succeeding edge are members of different components on $G'$.
\end{enumerate}

If such an edge is found, the procedure adds this edge to a set $C_i$, and continues to add all succeeding edges in the list to $C_i$, provided that for each edge conditions 1 and 2 hold, and the succeeding edge is not a member of a component on $G'$ that already has an edge in $C_i$. Once there are no more valid edges to add to $C_i$, the procedure resumes its search through the remaining edges in the list to find a new edge that meets conditions 1 and 2, and can therefore start a new set $C_{i+1}$. The procedure ends when the penultimate edge in the list has been examined.

The sets $C_1, C_2, ..., C_k$ created during this process are then analysed. In the event that the procedure has been unable to produce a single set $C$, the components $A_1,..., A_{\lambda}$ of $G'$ cannot be combined, and therefore no solution exists. If there exists a set $C_i$ such that $|C_i| = \lambda$, then all components on $G'$ can be combined using the following ``connecting method'':

Starting from the first edge in $C_i$, add the edge from $E$ connecting the smaller vertex of each edge to the larger vertex of the succeeding edge to $Q$. For the final edge in $C_i$, add the edge from $E$ that connects its smaller vertex to the larger vertex of the first edge in $C_i$ to $Q$. Finally, any edges that appear in both $C_i$ and $Q$ are removed from $Q$, such that $C_i \cap Q = \emptyset$, and $Q$ returns to being a perfect matching. $G'$ then comprises a single alternating Hamiltonian cycle, and thus the algorithm has found a solution to the problem instance.


Else, if $k > 1$, it may be that multiple sets are required to combine the components of $G'$. Let $\mathcal{C}$ be a set that contains the sets $C_i$ needed to connect the components of $G'$ together, and let $\Lambda$ be a set that records the components covered by the sets in $\mathcal{C}$. Start by adding the components $A_i$ that each edge in $C_1$ is a member of to $\Lambda$, and then add $C_1$ to $\mathcal{C}$. At this stage, $|\Lambda| = |\mathcal{C}|$. Proceeding through the remaining sets $C_2, C_3, ..., C_k$, find a set $C_i$ that contains at least one edge that is a member of a component in $\Lambda$, and at least one edge that is a member of a component not yet recorded in $\Lambda$. Append the component that each edge in this set $C_i$ is a member of to $\Lambda$, provided the component is not already an element of $\Lambda$, then add $C_i$ to $\mathcal{C}$. Continue in this manner until the final set $C_k$ has been assessed. If $|\Lambda| = \lambda$, i.e. all components of $G'$ are elements of $\Lambda$, then the sets in $\mathcal{C}$ are able to combine the components of $G'$ to form a single alternating Hamiltonian cycle. This is achieved by repeating the connecting method described previously on every set $C_i \in \mathcal{C}$.


This algorithm, which will be referred to as the Alternating Hamiltonian Construction Algorithm (AHCA) is able to deduce whether a feasible solution $\mathcal{T}$ exists for any given instance $\mathcal{M}$ of the CPOP. If a solution does indeed exist, AHCA can produce this solution in at most $O(n^2)$ time.

\section{Solving the SCSPP/Heuristics}
\label{sec:scsppsoln}

Indeed, the SCSPP is a generalisation of the classical one-dimensional bin-packing problem (BPP), where in the latter problem, the minimum scoring distance $\tau$ can be said to be equal to zero. It follows that the SCSPP is at least as hard as the BPP, which is known to be NP-hard (\citealp{garey1979}), and so (under the assumption that $P \neq NP$) there is no known algorithm that is able to find an optimal solution for every instance of the SCSPP in polynomial time. Instead, heuristics can be used to find near-optimal solutions in a shorter amount of time. One example is the greedy heuristic known as first-fit (FF), an online algorithm that places each item (in some arbitrary order) onto the lowest-indexed strip such that the capacity of the strip is not exceeded. It has been shown that by arranging the items in decreasing order of size, a packing heuristic will produce a solution that is closer to the optimal than if the items are arranged in any other order (\citealp{johnson1974fast}). Applying FF onto items sorted in this manner yields the well-known first-fit decreasing (FFD) heuristic.

An optimal solution of an instance of the SCSPP is a solution that consists of the fewest number of strips, $k$. A simple lower bound for $k$ that can be computed in $O(n)$ time is given by

\begin{equation}
	\label{eqn:lowerbound}
	k \geq \ceil*{\frac{\sum_{i=1}^{n}w_i}{l}}
\end{equation}

where $n$ is the number of items in the set $\mathcal{I}$, $w_i$ the width of each item $i \in \mathcal{I}$, and $l$ is the length of the strips (\citealp{martello1990b}). In 1973, \citeauthor{johnson1973} showed that FFD is guaranteed to return a solution that uses no more than $\frac{11}{9}k + 4$ strips. More recently, \cite{dosa2007} has proven that the worst case for FFD is in fact $\frac{11}{9}k + \frac{6}{9}$, and that this bound is tight. Due to the initial sorting of the $n$ items in non-increasing order of sizes, the time complexity of FFD is $O(n\log n)$.


\subsection{FFD Approx}
\label{sec:ffdapprox}
The first heuristic to be discussed is the Score-Constrainted First-Fit Decreasing (SCFFD) heuristic, which is an extension of the original FFD. The items in $\mathcal{I}$ are, as usual, sorted in non-increasing order of widths, with ties broken by selecting the item with the smallest score width. Then, for each item $i \in \mathcal{I}$, SCFFD finds the lowest-indexed strip $S_j$ that can accommodate $i$ without being overfilled. If this strip $S_j$ is empty, place $i$ onto $S_j$ in a regular orientation, and add $S_j$ to the solution set $\mathcal{S}$. However, if $S_j$ is not empty, i.e. $S_j \in \mathcal{S}$, the algorithm checks if the vicinal sum constraint is met between the right-most score width on $S_j$ and one of the two score widths on $i$, assessing the smaller score width $a_i$ first. In either case, if the constraint is met, $i$ is placed accordingly onto $S_j$. Indeed, an item $i$ may not fulfil the vicinal sum constraint in any orientation, in which case SCFFD finds the next lowest-indexed strip and attempts to pack $i$ again. This process continues until every item $i \in \mathcal{I}$ has been placed onto a strip $S_j \in \mathcal{S}$.

\subsection{Smallest First}
\label{sec:ffdsmall}
This next heuristic focusses on packing each strip in turn, rather than each item as in the previous heuristic. Each new strip $S_j$ is initialised by placing the item $i \in \mathcal{I}'$ with the smallest score width onto $S_j$ in a regular position, where $\mathcal{I}'$ is the set of items that have not yet been packed onto a strip. The heuristic then continues to pack items onto $S_j$, by choosing an item in $\mathcal{I}'$ with the smallest score width that fulfils the vicinal sum constraint when placed alongside the right-most score width on $S_j$, and that does not exceed $S_j$'s capacity. Once there are no more items in $\mathcal{I}'$ that can be feasibly packed onto $S_j$, the heuristic adds $S_j$ to the solution set $\mathcal{S}$, and the next new strip $S_{j+1}$ is initialised. When $|\mathcal{I}'| = 0$, all items have been packed onto a strip, and the process terminates.

\subsection{FFD Exact}
\label{sec:ffdexact}
The last heuristic in this paper is formed by integrating the Alternating Hamiltonian Construction Algorithm (AHCA) detailed in Section ref{} with the classical FFD heuristic. Again, all items in $\mathcal{I}$ are sorted into non-increasing order of widths, with ties broken by choosing the item with the smallest score width. For each item $i \in \mathcal{I}$, the heuristic finds the lowest-indexed feasible strip $S_j$. As in the first heuristic, if $S_j$ is empty, $i$ is placed in a regular orientation onto $S_j$, and $S_j$ is then added to the solution set $\mathcal{S}$. Unlike the first heuristic, however, if $S_j$ is not empty ($S_j \in \mathcal{S}$), AHCA is applied on all items in $S_j$ and the current item $i$. If AHCA is successful, the items currently on $S_j$ are replaced by the new arrangement, which of course includes $i$. In the event that AHCA is unable to produce a solution, $S_j$ remains intact, and the heuristic attempts to pack $i$ onto the next lowest-indexed feasible strip.



\bibliographystyle{dcu}
\bibliography{includes/bibliography}

\end{document}