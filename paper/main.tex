\documentclass[oribibl]{llncs}

%---- PREAMBLE ----%

% Allows separate files to be used in the main file
\usepackage{subfiles}
\usepackage{natbib}
\bibpunct{(}{)}{;}{a}{,}{,}

% Algorithms
\usepackage{algorithm}
\usepackage{algpseudocode}
\renewcommand{\algorithmicrequire}{\textbf{Input:}}
\renewcommand{\algorithmicensure}{\textbf{Output:}}
\newcommand{\IndStatex}[1][1]{\Statex\hspace{6mm}}
\newcommand{\IndIndStatex}[1][1]{\Statex\hspace{12mm}}
\newcommand{\IndIndIndStatex}[1][1]{\Statex\hspace{18mm}}
\newcommand{\IndIndIndIndStatex}[1][1]{\Statex\hspace{24mm}}
\newcommand{\IndIndIndIndIndStatex}[1][1]{\Statex\hspace{30mm}}
\newcommand{\IndIndIndIndIndIndStatex}[1][1]{\Statex\hspace{36mm}}

% Contains advanced math extensions
\usepackage{amsmath}

% Introduces the *proof* environment and the \theoremstyle command
%\usepackage{amsthm}

% Adds new symbols to be used in math mode, e.g. \mathbb
\usepackage{amssymb}

% To declare multiple authors
\usepackage{authblk}

% Provides extra comands as well as optimisation for producing tables
\usepackage{booktabs}

% Allows customisation of appearance and placements for figures/tables etc.
\usepackage{caption}

% Adds support for arbitrarily-deep nested lists
\usepackage{enumitem}

% Improves the interface for defining floating objects such as figures/tables
\usepackage{float}

\usepackage{fullpage}

% For easy management of document margins and the document page size
%\usepackage{geometry}

% Allows insertion of graphic files within a document
\usepackage{graphicx}

% Manage links within the document or to any URL when you compile in PDF
\usepackage{hyperref} 
\usepackage{xcolor}
\hypersetup{
	colorlinks,
	linkcolor={red!50!black},
	citecolor={green!30!black},
	urlcolor={blue!80!black}
}

% Successor of amsmath
\usepackage{mathtools}

% No indentation, space between paragraphs
\usepackage{parskip}

%Include standalone .tex files
\usepackage{standalone}

% Define multiple floats (figures/tables) within one environment with individual captions 1a, 1b etc
\usepackage{subcaption}

\usepackage{tikz}

\usepackage{wrapfig}

\DeclarePairedDelimiter{\floor}{\lfloor}{\rfloor}
\DeclarePairedDelimiter{\ceil}{\lceil}{\rceil}

%Theorem style
%\theoremstyle{plain}% default
%\newtheorem{theorem}{Theorem}[section]
%\newtheorem{corollary}{Corollary}[theorem]

%\theoremstyle{definition}
%\newtheorem{defn}{Definition}[section]
%\newtheorem{proposition}{Proposition}[defn]
%\newtheorem{exmp}{Example}[section]

\title{Heuristics for the Score-Constrained Bin-Packing Problem}
\author{Asyl L. Hawa \and Rhyd M. R. Lewis \and Jonathan M. Thompson}
\institute{School of Mathematics, Cardiff University, Senghennydd Road, Cardiff, UK, CF24 4AG}
\date{\today}

\begin{document}
\maketitle

\begin{abstract}
	
\end{abstract}

\section{Introduction}
\label{sec:intro}

Modification/adaption of the Pair Sequencing Problem (PSP) as seen in \cite{lewis2017}:

\begin{definition}
	\label{defn:psp}
	Let $\mathcal{P}$ be a multiset of unordered pairs $\mathcal{P} = \{\{x_1, y_1\}, \{x_2,y_2\},...,\{x_n,y_n\}\}$, $x_i, y_i \in \mathbb{Z}^{+}$ for $i \in \{1,2,...,n\}$, and let $\mathcal{X}$ be an ordering of the elements on $\mathcal{P}$ in which each element is also expressed as an ordered pair. The \textcolor{RoyalBlue}{constrained} PSP involves finding a solution $\mathcal{X}$ such that
	\begin{equation*}
		\textup{\textbf{rhs}}(i) + \textup{\textbf{lhs}}(i+1) \geq \tau \hspace{5mm} \forall \hspace{1mm} i \in \{1,2,..., n-1\}
	\end{equation*}
	where \textup{\textbf{lhs}($i$)} and \textup{\textbf{rhs}($i$)} denote the values on the left- and right-hand sides of the $i$th ordered pair in $\mathcal{X}$, and $\tau \in \mathbb{R}^{+}$ is a fixed value.	
\end{definition}

For example, if $\mathcal{P} = \{\{1,2\}, \{1,5\}, \{2,4\}, \{3,4\}, \{4,5\}\}$ and $\tau = 7$, a solution is $\mathcal{X} = \langle(1,2), (5,4), (3,4), (4,2), (5,1) \rangle$.














\bibliographystyle{dcu}
\bibliography{includes/bibliography}

\end{document}