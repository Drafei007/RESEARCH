\documentclass[oribibl]{llncs}

%---- PREAMBLE ----%

% Allows separate files to be used in the main file
\usepackage{subfiles}
\usepackage{natbib}
\bibpunct{(}{)}{;}{a}{,}{,}

% Algorithms
\usepackage{algorithm}
\usepackage{algpseudocode}
\renewcommand{\algorithmicrequire}{\textbf{Input:}}
\renewcommand{\algorithmicensure}{\textbf{Output:}}
\newcommand{\IndStatex}[1][1]{\Statex\hspace{6mm}}
\newcommand{\IndIndStatex}[1][1]{\Statex\hspace{12mm}}
\newcommand{\IndIndIndStatex}[1][1]{\Statex\hspace{18mm}}
\newcommand{\IndIndIndIndStatex}[1][1]{\Statex\hspace{24mm}}
\newcommand{\IndIndIndIndIndStatex}[1][1]{\Statex\hspace{30mm}}
\newcommand{\IndIndIndIndIndIndStatex}[1][1]{\Statex\hspace{36mm}}

% Contains advanced math extensions
\usepackage{amsmath}

% Introduces the *proof* environment and the \theoremstyle command
%\usepackage{amsthm}

% Adds new symbols to be used in math mode, e.g. \mathbb
\usepackage{amssymb}

% To declare multiple authors
\usepackage{authblk}

% Provides extra comands as well as optimisation for producing tables
\usepackage{booktabs}

% Allows customisation of appearance and placements for figures/tables etc.
\usepackage{caption}

% Adds support for arbitrarily-deep nested lists
\usepackage{enumitem}

% Improves the interface for defining floating objects such as figures/tables
\usepackage{float}

\usepackage{fullpage}

% For easy management of document margins and the document page size
%\usepackage{geometry}

% Allows insertion of graphic files within a document
\usepackage{graphicx}

% Manage links within the document or to any URL when you compile in PDF
\usepackage{hyperref} 
\usepackage{xcolor}
\hypersetup{
	colorlinks,
	linkcolor={red!50!black},
	citecolor={green!30!black},
	urlcolor={blue!80!black}
}

% Successor of amsmath
\usepackage{mathtools}

% No indentation, space between paragraphs
\usepackage{parskip}

%Include standalone .tex files
\usepackage{standalone}

% Define multiple floats (figures/tables) within one environment with individual captions 1a, 1b etc
\usepackage{subcaption}

\usepackage{tikz}

\usepackage{wrapfig}

\DeclarePairedDelimiter{\floor}{\lfloor}{\rfloor}
\DeclarePairedDelimiter{\ceil}{\lceil}{\rceil}

%Theorem style
%\theoremstyle{plain}% default
%\newtheorem{theorem}{Theorem}[section]
%\newtheorem{corollary}{Corollary}[theorem]

%\theoremstyle{definition}
%\newtheorem{defn}{Definition}[section]
%\newtheorem{proposition}{Proposition}[defn]
%\newtheorem{exmp}{Example}[section]

\title{Heuristics for the Score-Constrained Bin-Packing Problem}
\author{Asyl L. Hawa \and Rhyd M. R. Lewis \and Jonathan M. Thompson}
\institute{School of Mathematics, Cardiff University, Senghennydd Road, Cardiff, UK, CF24 4AG}
\date{\today}

\begin{document}
\maketitle

\begin{abstract}
	
\end{abstract}

\section{Introduction}
\label{sec:intro}

\begin{definition}
	\label{defn:cpop}
	Let $\mathcal{M}$ be a multiset of unordered pairs of positive integers $\mathcal{M} = \{\{a_1, b_1\}, \{a_2,b_2\},...,\{a_n,b_n\}\}$, and let $\mathcal{T}$ be an ordering of the elements of $\mathcal{M}$ such that each element is a tuple. The Constrained Pair Ordering Problem (CPOP) consists of finding a solution $\mathcal{T}$ such that, given a fixed value $\tau \in \mathbb{R}^{+},$
	\begin{equation}
		\label{eqn:cpop}
		\textup{\textbf{rhs}}(i) + \textup{\textbf{lhs}}(i+1) \geq \tau \hspace{5mm} \forall \hspace{1mm} i \in \{1,2,..., n-1\},
	\end{equation}
	where \textup{\textbf{lhs}($i$)} and \textup{\textbf{rhs}($i$)} denote the left- and right-hand values of the $i$th tuple.	
\end{definition}

For example, given the multiset $\mathcal{M} = \{\{1,2\}, \{1,5\}, \{2,4\}, \{3,4\}, \{4,5\}\}$ and $\tau = 7$, one possible feasible solution is $\mathcal{T} = \langle(1,2), (5,4), (3,4), (4,2), (5,1) \rangle$.

\textcolor{OrangeRed}{Model the problem using a graph.}

Let $\mathcal{M}$ be a multiset as described in Definition \ref{defn:cpop}, with an additional unordered pair consisting of ``dominating'' values equal to $\tau$. Two values that make up a pair in $\mathcal{M}$ are referred to as ``partners''. Then, $G(V, P \cup E)$ is an undirected graph defined by a vertex set $V = \{v_1, v_2, ...v_{2n+2}\}$, where each vertex is assigned a value from $\mathcal{M}$ such that $v_1 \leq v_2 \leq ... \leq v_{2n+2}$. $G$ comprises two distinct edge sets: $P$, which contains edges between vertices that are partners, and $E$, which consists of edges between vertices that fulfil the \textit{joint sum constraint}. That is, for two distinct vertices $v_i, v_j$,

\begin{equation}
	\label{eqn:jointsum}
	v_i + v_j \geq \tau,
\end{equation}

provided $\{v_i, v_j\} \notin P$. Observe that $P$ is in fact a perfect matching, with $|P| = |\mathcal{M}| = n+1$, and that $P \cap E = \emptyset$.

If $G(V, P\cup E)$ features an alternating Hamiltonian cycle, then a feasible solution $\mathcal{T}$ exists.

\begin{definition}
	\label{defn:althamcycle}
	Let $G(V, P\cup E)$ be an undirected graph, where each edge is a member of one of two sets, $P$ or $E$. $G$ contains an alternating Hamiltonian cycle if there exists a Hamiltonian cycle such successive edges alternate between sets $P$ and $E$. For example, if $\{v_{i-1}, v_i\} \in P$, then $\{v_i, v_{i+1}\} \in E$, or vice versa.
\end{definition}

\begin{theorem}
	\label{thm:cpopsolnalt}
	There exists a feasible solution $\mathcal{T}$ to an instance $\mathcal{M}$ of the CPOP if and only if its corresponding graph $G(V, P\cup E)$ contains an alternating Hamiltonian cycle.
\end{theorem}
\begin{proof}
	To be completed. \qed
\end{proof}









































%========================================================================%


\begin{comment}
The PSP can be seen in a problem related to the cutting-stock problem, first described by \cite{goulimis2004}. Let $\mathcal{I}$ be a set of rectangular items of equal heights, made from corrugated cardboard. Each item $i \in \mathcal{I}$ has width $w_i \in \mathbb{Z}^{+}$ and possess two vertical score lines. A pair of knives mounted on a bar will cut along these score lines to allow the items to be folded with ease. The distances between each edge of an item and the nearest score line are the score widths, $a_i, b_i \in \mathbb{Z}^{+}$, assigned such that $a_i \leq b_i$. As illustrated in Figure \ref{fig:boxesknife}, the knives cut along score lines from two adjacent items simultaneously. By design, the knives cannot be placed too close to one another - the distance between the knives must exceed a fixed minimum $\tau \in \mathbb{R}^{+}$ (around 70mm in industry). In order for the knives to be able to cut the items in the correct locations, two adajcent score widths, for example $a_i$ and $b_{i+1}$, must fulfil the \textit{minimum total score width constraint}

\begin{equation*}
a_i + b_{i+1} \geq \tau.
\end{equation*}

That is, the sum of two adjacent score widths must exceed the \textit{minimum total score width} $\tau$. Therefore, the problem involves finding a feasible arrangement of the items such that between each pair of items the minimum total score width constraint is met.


\begin{figure}[H]	
\centering
\includestandalone[width=0.4\textwidth]{figures/boxesknifeannotate}
\caption{\textcolor{red}{Individual boxes.}}	
\label{fig:boxesknife}
\end{figure}

Since the score widths on each item $i \in \mathcal{I}$ are not necessarily equal, they can be aligned in two distinct orientations: ``regular'', denoted by $(a_i, b_i)$, and ``rotated'', denoted by $(b_i, a_i)$, where the smaller of the two score widths $a_i$ is on the left- and right-hand side of the item respectively.

This can be seen as a PSP, where the values $x_i$ and $y_i$ are the score widths, and a feasible solution $\mathcal{X}$ consists of an arrangement of items in some order such that $\textup{\textbf{rhs}}(i) + \textup{\textbf{lhs}}(i+1) \geq \tau \forall  i \in \{1,2,..., |\mathcal{I}|-1\}$.

There exists a polynomial-time algorithm that is able to recognise whether a particular instance of this problem is feasible (\citealp{becker2010}).

\end{comment}



\bibliographystyle{dcu}
\bibliography{includes/bibliography}

\end{document}